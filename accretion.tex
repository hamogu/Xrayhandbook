


\section{Accretion \label{sect:accretion}}

Accretion is the defining characteristic of young stars. It is through accretion that they build up mass, it is through accretion that they accumulate angular momentum, and it is probably through magnetic connection between disk and accretion column that at least some part of their angular momentum is lost. Once accretions stops, the mass, chemical composition, and angular momentum of a star continue to evolve through winds, but on much longer time scales.

Today it is widely accepted that the accretion in T Tauri stars is magnetically funneled \citep{Hartmann_2016}. The accretion disk does not reach down to the stellar surface. Instead, it is truncated at a few stellar radii close to the radius where the disk co-rotates with the central star. While magnetic fields of young stars can be complex, at large distances the field will be dominated by a dipole component. Magnetic field lines couple the star with the inner disk and disk material, ionized by the UV and X-ray radiation from the star, is forced to follow the field lines. As it falls in, gravity accelerates the matter to almost free-fall velocity until it hits the surface where a strong shock develops \cite{Shu_1994}. 

The shock is thought to produce the observed X-ray emission related to the accretion process. Therefore, we first dicuss the accretion stream and the locations of the footpoints. Then, we describe simple-1D accretion shocks  before we turn to more detailed observations and models.

\subsection{The accretion stream and its foot points}
\label{sect:accretionsrteam}
The accretion stream is initially cool ($\log T\sim3-4$); it heats up as mass accelerates and comes close to the star. The most prominent tracers are the strong and complex hydrogen emission lines. In particular H$\alpha$ is usually optically thick and often shows red-shifted absorption components compatible with free-fall velocity \cite[e.g.][]{2000AJ....119.1881A}, and varies over time scales of hours \cite{dupree_2012}. Many emission lines do not vary with the stellar rotation period, indicating that it is the inner disk, not the anchor point on the stellar surface, that controls the geometry of accretion \cite{2021A&A...649A..68S}.



Zeeman-Doppler imaging can reveal the structure of the magnetic field and, using certain assumptions, those fields can be extrapolated out to the inner disk edge. The dipole component usually dominates the field at large radii and is thus most important for coupling the star with the accretion disk\footnote{There seems to be an evolution of magnetic field strength and geometry, where the strength of the dipole component decreases with the depth of the convective zone \cite{2012ApJ...755...97G,2019A&A...622A..72V}, but even if higher magnetic moments dominate at the stellar, the dipole typically dominates at the inner disk edge.}; the accretion funnel therefore follows the dipolar field lines. On the stellar surface itself, Doppler imaging can locate the position of the accretion funnels which are often found near the pole, e.g.\ in BP Tau \cite{2008MNRAS.386.1234D} or V2129 Oph \cite{2011A&A...530A...1A}, but sometimes at lower latitudes as in V2247 Oph \cite{2010MNRAS.402.1426D}. 
Simulations can reproduce the analytical model of accretion foot points near the pole in the dipolar field, but they also point to more complex geometries when disk, stellar rotation, and stellar magnetic field are not aligned (Fig.~\ref{fig:romanova}).

Accretion is a rather dynamic process. As the star and the disk rotate, and the magnetic field and the disk structure evolve, the accretion geometry and the accretion rate can change on time scales as short as minutes or as long as centuries; accretion can also switch off temporarily or permanently, as the star looses its disk. Nevertheless, the basics of the  accretion process remain similar as long as the star accretes from the disk, and we expect that X-rays are produced by the same processes throughout.

\begin{figure}[t]
\centering
\includegraphics[width=5cm]{figs/Romanova2021fig8-panel.png}
\caption{Simulation of accretion on an inclined dipole. Mass flows towards the magnetic pole \cite{2021MNRAS.506..372R} \textcolor{blue}{need to ask for permission} \label{fig:romanova}}
\end{figure}





\subsection{X-ray signatures of the accretion shock}
\label{sect:accretionobs}
Any accretion generated X-ray emission in young stars must be seperated from the ubiquitous X-ray emission by coronal activity in low-mass stars. Young stars rotate faster and thus have more magnetic activity and stronger coronal X-rays than older stars such as our Sun. That is why accretion signatures are hard to find in broad-band X-ray spectra  with their limited spectral resolving power. To the contrary, one rather finds an inverse correlation between X-ray flux  accretion rate \cite{2005ApJS..160..401P, Schneider_2018}. However, this does not contradict the idea that accretion shocks generate soft X-rays as models show that the shock would contribute only below 1~keV \cite{1999AstL...25..430L} and those soft X-rays are easily absorbed by circumstellar material or the remnants of the star forming cloud.
% , and soft X-rays from the accretion shock may not be powerful enough to make up for other effects that might reduce coronal activity, such as the lower rotation rate compared to disk-less stars of the same age \textcolor{blue}{reference here}.

Therefore, additional X-ray diagnostics are needed to test the existence of accretion generated X-ray emission. High-resolution grating spectroscopy is among the most important ones, as it allows one to measure the density of the emitting plasma  from line ratios in the O~{\sc vii} and Ne~{\sc ix} triplets. XMM-Newton and Chandra carry X-ray gratings with sufficient spectral resolution to separate the three lines of the density sensitive triplets: A resonance line ($r$), an intercombination line ($i$), and a forbidden line ($f$). In collisionally excited plasma, the $f$ line is typically stronger than the $i$ line, but collisions in high-density plasma or strong UV fields (relevant in A or B stars, but not in the lower-mass classical T Tauri stars) can excite an electron from the upper level of the $f$ line to the upper level of the $i$ line. A low $f/i$ ratio is thus a sign of high densities in the emission region. Because coronal emission typically comes from low density plasma \citep[$n_e\lesssim10^{10}$\,cm$^{-3}$, e.g.,][]{Ness_2002}, a X-ray emission from a high density likely originates behind the shock front of an accretion shock, where densities are thought to be much higher than in the corona ($\gtrsim10^{13}$\,cm$^{-3}$). Figure~\ref{fig:softexcess} (left panel) shows examples for three CTTS which all show $f/i < 1$ compared to a typical main-sequence star with $f/i\sim 4$. A triplet indicative of high densities was first seen in TW~Hya \cite{Kastner_2002}, but the same pattern has since been confirmed in a number of CTTS. Low $f/i$-ratios are therefore strong evidence for accretion generated X-rays.

\begin{figure}[t]
\centering
\includegraphics[width=0.49\textwidth]{figs/o7f2i.png}
\includegraphics[width=0.49\textwidth]{figs/o72o8.png}
\caption{Signatures of accretion in classical T Tauri stars from high-resolution grating spectroscopy. \emph{Left:} Density-sensitive O~{\sc vii} triplet. Capella is a main-sequence star with an $f/i\sim 4$, while the other three sources are examples of CTTS with $f/i < 1$. All lines are unresolved, they appear wider in BP~Tau because the data is taken with a lower-resolution spectrograph (Chandra/LETG, while the other three are Chandra/HETG). \emph{Right:} Ratio of O~{\sc viii} to O~{\sc vii} line flux compared to the total flux in oxygen lines. All acrreting sources are well offset from the main-squence stars, indicating additional soft emission plasma. Modified from Ref.~ \cite{2013ApJ...771...70G} (see there for data sources) \label{fig:softexcess}}
\end{figure}


Another feature seen in X-rays that sets accreting T~Tauri stars (CTTS) apart from their non-accreting sibblings (WTTS) is a lower O~{\sc viii}/O~{\sc vii}-ratio (see Fig.~\ref{fig:softexcess} right), which indicates that the X-ray emission in CTTS is dominated by cooler plasma compared to non-accreting stars of comparable luminosity \cite{2007A&A...473..229R,2007A&A...474L..25G}. 
Looking from a different angle, one may also assume that CTTS have an intrinsic (coronal) O~{\sc viii}/O~{\sc vii}-ratio  in line line with WTTS or main-sequence stars, where brighter coronae are also hotter. This view implies that CTTS have additional cool plasma compared to normal corona, and this additional plasma may be accretion-powered as the expected temperature of the post-shock plamaa radiates strongly in O~{\sc vii} and not in O~{\sc viii}.

A third observation is that CTTS usually show abundances that are Ne enhanced and Fe depleted compared to solar abundances \citep{Stelzer_2004}. This pattern is seen in active stars, where it is attributed to element separation in the corona due to different values of the first ionization potential (FIP) of ions. Neon has a particularly high FIP and iron a particularly low one, so the pattern observed is called IFIP (inverse FIP). However, CTTS are cooler than main-sequence stars of comparable luminosity and so alternative scenarios have been discussed \citep{Drake_2005}. Since accretion comes from the inner edge of the accretion disk, it is possible that disk processes separate elements. If gain-forming elements condense into grains, pebbles, and proto-planets, the material on the inner disk edge might be depleted of Fe, Si, and similar metals, while noble gases stay in the gas phase and are thus preferentially accreted. 

Fourth, X-ray grating spectra of TW Hya have sufficient S/N to determine  line shifts for some strong emision lines. The lines from cooler plasma show a $38.3 \pm 5.1$~km/s shift wrt the hotter, coronal lines \cite{2017A&A...607A..14A}. And while this indicates different origins for the cool and hot plasma, it is also much less than the free-fall velocity. This implies that we must see the shock almost perpendicular to the line-of-sight. TW~Hya is observed close to pole-on, so the accretion shock (s) seen in X-rays must the located in the equatorial region while spectropolarimetric observations suggest that accretion footpoints are mostly close to the pole \cite{Donati_2011}.


\subsection{Physics of accretion in 1D}
\label{sect:accretionphysics}
As matter impacts on the photosphere an X-ray emitting shock forms. Depending on the height and geometry, those X-rays may or may not be visible, but the shock certainly heats the surrounding photosphere, which causes bright UV emission and optical veiling (a strong continuum that makes phtotospheric emission lines appear weaker than in a non-accretion star).

Many aspects of the accretion physics can be described in a 1D model where all mass motion happens parallel to the magnetic field. In the next few sub-sections we review some of the basic physics of accretion columns and accretion shocks. While modern models go far beyond such a simple prescription, the foundation of all accretion shock models is to convert the gravitational energy in the disk to kinetic energy of the free-falling gas, which in turn gets turned into heat and radiation in the accretion shock.

The free fall velocity $v_{\textnormal{free}}$ of material coming from an inner disk radius of $R_\mathrm{in}$ onto a star with mass $M_*$ and radius $R_*$ is
\begin{equation}
v_{\textnormal{free}} = \sqrt{{2GM_*} \left(\frac{1}{R_*} - \frac{1}{R_\mathrm{in}}\right)} \approx 620 \sqrt{\frac{M_*}{M_\odot}}\sqrt{\frac{R_\odot}{R_*}} \frac{\textnormal{km}}{\textnormal{s}}\ \label{eqn:freefall}
\end{equation}
where $G$ is the gravitational constant. The inner radius is typically a few stellar radii and thus $v_\textnormal{ff}$ is very close to infall from infinity.


\subsubsection{The shock front}
We concentrate on stationary shocks and ignore all turbulent fluxes. In the shock front, ions and electrons are heated differently, but they remain strongly coupled and reach the same temperatures with in a few mean-free path lengths - a region so thin that it is justified to treat them as a single fluid.

Somewhere along the accretion column, a shock forms when the forward ram pressure becomes comparable to the pressure of the underlying material. The shock front itself is very thin, only of the order of a few mean free paths \cite{raizerzeldovich}. Therefore it can be treated as a mathematical discontinuity described by the Rankine-Hugoniot jump-conditions \cite[][chap.~7, \S~15]{raizerzeldovich}; in the shock the super-sonic infall velocity is converted mostly into thermal energy. Since we assume the direction of flow parallel to the magnetic field, the Lorentz force does not influence the dynamics. Marking the state in front of the shock front by the index 0, that behind the shock by index 1, the Rankine-Hugoniot conditions become
\begin{eqnarray}
\rho_0 v_0 &=& \rho_1 v_1 \label{RH1}\\
P_0+\rho_0 v_0^2 &=& P_1+\rho_1 v_1^2 \label{RH2}\\
\frac{5 P_0}{2\rho_0}+\frac{v_0^2}{2}&=&\frac{5 P_1}{2\rho_1}+\frac{v_1^2}{2} \ ,\label{RH3}
\end{eqnarray}
where $v$ is the velocity, $\rho$ the total mass density of the gas and $P$ its pressure.

From the jump conditions, the shocks will heat gas to a temperature
\begin{equation}
kT \simeq \frac{3}{16}\mu m_p v^{2} \approx 0.3\,{\rm keV}\left(\frac{v}{500\,{\rm km/s}}\right)^{2} \approx3.5\times10^6\,{\rm K} \left(\frac{v}{500\,{\rm km/s}}\right)^{2},
\label{eqn:Tshock}
\end{equation}
where $\mu$ is the dimensionless atomic weight.

\subsubsection{Structure of the post-shock region}

In the post-shock region the gas emits radiation and cools down, so the energy of the gas is no longer conserved.  However, the particle number flux $j$ of ions (and atoms)
\begin{equation}j=nv\label{j_n}\end{equation}
is conserved, where $n$ is the ion/atom number density; the electron number density is denoted by $n_{\mathrm{e}}$. The total momentum flux $j_p$ is conserved, since we ignore the momentum loss by radiation:
\begin{eqnarray}
j_p&=&\mu m_{\mathrm{H}} n v^2+P \nonumber \\
   &=&\mu m_{\mathrm{H}} n v^2+nkT \label{j_p}
\end{eqnarray}
with $P$ is the thermodynamic pressure and $T$ the temperature; $m_{\mathrm{H}}$ denotes the mass of a hydrogen atom.

Let us next consider the energy balance in the post-shock region. In general,
\begin{equation} \label{tsminuspdvisdu} T d\Sigma -P dV=dU \end{equation}
where $\Sigma$ denotes the entropy and $U$ the internal energy of the plasma. The quantity $T d\Sigma=dQ$ denotes the heat flux through the boundaries of the system. Here, that is the energy loss $Q_{col}$ through collisions that excite higher electronic states, which will than decay through radiation.

Assuming that the shock location is stationary, we get $\frac{d}{dt}=\frac{\partial}{\partial t}+\frac{\partial z}{\partial t}\frac{\partial}{\partial z}=v\frac{\partial}{\partial z}$ depending on the location $z$, measured from the shock front inwards; differentiation with respect to $z$ will be indicated by $'$.
The internal energy $U$ is in this case the thermal energy $U=\frac{3}{2}kT$, the pressure $P$ can be rewritten using the equation of state. The specific volume $V$ is the inverse of the number density $V=\frac{1}{n}$.
It is convenient to write the electron number density as \mbox{$n_{\mathrm{e}}=x_e n$,} with $x_e$ denoting the number of electrons per heavy particle.
\begin{equation}
\label{energyelec}
v\left(\frac{3}{2}x_e k T_{\mathrm{e}}\right)'+v x_e n k T_{\mathrm{e}} \left(\frac{1}{n}\right)'=-Q_{col} x_e n,
\end{equation}

We now have $n$, $v$, and $T$ as variables and three hydrodynamic equations (\ref{j_n}, \ref{j_p}, and \ref{energyelec}), so the structure of the post-shock region can be calculated.

Simulations based on these or very similar formulas show that the accretion shock produces plasma matching the observed X-ray temperatures \citep{lamzin_1998} and the total energy in the accretion stream can be determined from fitting UV and optical spectra to determine the mass accretion rate \citep{calvet_1998}. Observations of the density-sensitive line ratios in He-like triplets such as in Fig~\ref{fig:softexcess} (left) can be explained by 1D models assuming a combination of accretion shock and coronal plasma \cite{Guenther_2007} and at the same time give a mass accretion rate that can be compared to observations of UV and optical tracers.


\subsection{Why we need to go beyond 1D models}
One dimensional models are successful in many aspects, but there are also observational and theoretical arguments suggesting that important physical processes are not captured in 1D. In deep Chandra observations of TW~Hya, densities can be measured in three density-sensitive triplets. The density is highest in Mg~{\sc xi} and lowest in O~{\sc vii}, although Mg~{\sc xi} is formed at a higher temperature \cite{Brickhouse_2010}. Because the post region is isobaric, one would expect Mg~{\sc xi} emission from a region directly behind the accretion shock and O~{\sc vii} from denser layers deeper down ($P\propto T$). One possible explanation is that the observed O~{\sc vii} emission is not from the accretion shock itself, but originates in hot material that escapes the accretion column to the side and is denser than a normal stellar corona, but not as dense as plasma behind the accretion shock. That in turn means that the total X-ray flux might not be a good measure of the total accretion rate.

This is corroborated by the surprising observation that X-ray determined mass accretion rates are very similar for most sources despite a difference in optically determined mass accretion rate by three orders of magnitude \cite{2011A&A...526A.104C} which can be explained if the accretion streams are not homogeneous structures but have a density profile and the inner layers either form shocks deeper in the atmosphere or simply have their X-ray emission reprocessed by the outer layers of the accretion stream \cite{2018A&A...618A..55S,2021Natur.597...41E}.

Time variability can give us another insight. In V4046~Sgr the emission lines from soft, presumably shock-heated, plasma have a period of exactly half the orbital period of the close binary. Together with Doppler-imaging this leads to the interpretation that we can observe the shock only when the accretion flow is perpendicular to the line of sight, while the accretion funnel blocks the view of the shocked region at other times \cite{2012ApJ...752..100A}. 
% Similarly, the accretion funnel has been observed to block the X-rays from AA~Tau for certain rotational phases. This includes coronal and accretion generated X-rays \cite{2007A&A...462L..41S,2007A&A...475..607G}.
There are several classes of variable pre-main sequence accretors, including FU~Ors and EX~Ors, where the accretion rate changes by orders of magnitude or stars where a change in the disk structure moves absorbing material into our light of sight, such as in RW~Aur or also in AA~Tau. However, those changes happen on much larger scales than the accretion shock itself and are not discussed here any further.

Similarly, improved models including LTE radiation transfer now show that the heated photosphere does not radiate as a simple black body in the optical and infrared \cite{Dodin_2012,Dodin_2013}, but also produces lines, which selectively fill in some photospheric absorption lines, possibly biasing accretion rate measurements based on optical veiling.
Similar optical depth effects also
have a non-negligible effect on the typical characteristics of the 
accretion dynamics, on the estimation of its X-ray surface luminosity 
\cite{Sa_2019}, on the heating of the accretion column \cite{1999AstL...25..430L,Costa_2017}, and the predicted lightcurve \cite{2021ApJ...908...16R}.
Thus, a correct description of the coupling between 
radiation and hydrodynamics is necessary to account for the effects of absorption and emission of radiation, which necessarily requires to account for the 3D structure of the accretion process, because photons and mass may travel differently through the domain.



\begin{figure}[t]
    \centering
    \includegraphics[width=11cm]{figs/colombo.png}
    \caption{Left: Adapted from \cite{Colombo_2019b}. Space-time map, in 
    logarithmic scale, of temperature for run RHD. The green region in the
bottom panel corresponds to the hottest part of the precursor. The gray
dotted lines in both panels mark the pre-impact position of the chromosphere. Right: Adapted from \cite{Colombo_2016}. Color maps of evolution of density (left half-panel) and temperature (right half-panel) of plasma for the general case of fragmented stream explored in \cite{Colombo_2016}. White lines represent magnetic field lines. Already asked permission to first author, need to email A\&A.}
    \label{fig:colombo}
\end{figure}


\subsection{The multi-D structure of the accretion shock}
Most observations of the structure of the accretion shock on young stars are spatially unresolved with the notable exception of Doppler-imaging, which can reveal the shock location on the surface. Yet, it would be very valuable to learn about the detailed 3D geometry and the time evolution of accretion shocks to interpret the unresolved data. For example, it is unclear if the accretion shocks form deeply in the photosphere where it is hidden from view or higher up in the accretion funnel. 
One approach to address this problem is to look for analogous situations in our Sun, where spatially resolved data in the UV and EUV, even if not in X-rays, is available with long time coverage and high cadence. One particularly interesting event happened on June, 7$^{th}$, 2011, when parts of an erupting filament fell back into the Sun \cite{2013Sci...341..251R} (see also \cite{Reale_2014}). The infall speed of up to 450~km/s was comparable to free-fall accretion onto T Tauri stars, but the accretion rate was obviously much lower. On the Sun, the initial infall triggered upflows that shocked with later fragments causing UV emission.




\begin{figure}
    \centering
    \includegraphics[width=11cm]{figs/column_sketch.pdf}
    \caption{Different structures of the accretion column.}
    \label{fig:column}
\end{figure}


To explain the discrepancies between the 1D models and observations, different scenarios have been proposed for the structure of accretion columns. The first models considered one homogeneous column with one density and a single infall velocity (see Fig.~\ref{fig:column}\,(a) and, e.g., \cite{Orlando_2010,Orlando_2013}). Numerical simulations based on this idea modelled a constant and uniform accretion stream that propagates along the magnetic field lines considering either uniform \cite{Orlando_2010} or nonuniform \citep{Orlando_2013} stellar magnetic fields at the impact region of accretion columns.
These studies concluded that the structure, dynamics and stability of the accretion shock strongly depend on the configuration and strength of the magnetic field, and thus on the plasma $\beta$ (defined as $\beta =$~gas pressure/magnetic pressure). In the case of shocks with high $\beta$ strong outflows of shock-heated material arise at the base of the accretion stream perturbing the stellar atmosphere and changing the dynamics of the accretion shock \cite{Orlando_2010}. Models show that at the border of the stream the plasma may not be efficiently confined by the magnetic field, where the shock-heated material can escape laterally (see \cite{Orlando_2010,Orlando_2013}).

More complex accretion streams have modelled with multi-D MHD simulations, too. For example, density stratification or clumpy structures formed by several blobs (Fig.~\ref{fig:column} (c) \& (d)). \cite{Matsakos_2013} investigated how perturbations in the accretion column (namely a clumpy stream and a oblique impact) can disrupt the shock structure. Different perturbations induced in the accretion stream seem to create a more inhomogeneous post-shock region with the post-shock region still strongly dependent on the plasma $\beta$.
More recent simulations considered the case of a randomly fragmented stream (see right panel in Fig.~\ref{fig:colombo}, see also \cite{Colombo_2016}), exploring different levels of stream fragmentation and accretion rates. In this case the accreting blobs impact onto the stellar chromosphere producing reverse shocks that propagate upflows through the unshocked fragmented stream. As a result, the structure of the post-shock region is very complex and consists of several knots and filaments of plasma with a wide range of velocities, densities, and temperatures (see right panel in Fig.~\ref{fig:colombo}).


Simulations now also include absorption by pre-shock material. An envelope of dense and cold material may develop around the shocked column, influencing the observability of the shock-heated plasma in the X-ray band \citep{Orlando_2013}. The X-ray emission from the shocked region can be heavily reduced due to the absorption by  optically thick material around the hot plasma, and the magnitude of this effect depends on the distribution of material along the line of sight (LoS) towards the shock \cite{Bonito_2014}. 
When using realistic radiative transport in the non-local thermodynamic equilibrium (non-LTE) regime, modelling of an accretion column impacting the surface of a CTTS show that $\approx 70$\% of the radiation emitted by the post-shock plasma is absorbed by the pre-shock accretion column forming a radiative precursor of the shock (see the green region in left panel of Fig.~\ref{fig:colombo} and \cite{Colombo_2019b}).                  

In addition to MHD models, laboratory experiments are now available, which can be scaled to accretion shocks on CTTS. The MHD equations are invariant under certain transformations and the magnetic fields, densities, temperatures, and other hydrodynamical parameters of the experiments are such that they scale to the stellar case. 
One example of such an experiment consisted of a collimated narrow laser-produced plasma stream that propagates along a strong external magnetic field, very similar to accretion on CTTS \cite{Revet_2017}. The laboratory experiment shows
that, upon impact, the plasma is ejected laterally from the accretion shock and then refocused by the magnetic field towards the incoming stream, forming a shell that envelops the shocked core \cite{Revet_2017}. This hot shell provides an additional absorber that reduces the observable  X-ray emission. Also,  experiments with accretion streams impacting a tilted surface have been performed, which correspond to accretion stream along complex magnetic field structures. The resulting plasma flows is highly asymmetric and a large amount of plasma escapes laterally from the accretion flow, showing poor confinement of the accreted material and reduced heating compared to the normal
incidence case \cite{Burdonov_2020}.

In summary, laboratory experiments, MHD numerical simulations, and the solar analogy, point to a significantly more complex picture than  simple 1D accretion shocks.
Multi-D models support the scenario of complex structure of the accretion shock probably developed as a result of a density structured or fragmented accretion stream, where the strength and configuration of the magnetic field also plays an important role. However, all the multi-D models presented in this section assume that the plasma is optically thin in the whole domain due to the high computational cost of including the radiative transfer in the model. Taking into account the absorption by optically thick material and the effect of radiative heating in future models would allow to investigate more deeply the origin of the emission arising from the accretion shock.

\subsection{Time variability Accretion outbursts}
In the previous sections, we concentrated on CTTS with their well-ordered magnetically-funneled accretion streams. There are other classes of pre-main sequence accretors, but those are rare and hard to observe. One group are accretion outbursts, where the accretion rate suddenly jumps by several orders of magnitude and then decays back over time scales from months (EX~Or objects) to decades or centuries (FU~Or objects) \cite{2014prpl.conf..387A}. EX~Or outbursts are not only shorter, they can also be observed to be repetitive. These eruptive stars show bright and hot coronal emission, possibly because the coronal structures are compressed with a higher density or because having a bright corona in the first place somehow triggers the accretion burst \cite{2011ApJ...741...83T,2019ApJ...883..117K}. Most sources are highly absorbed, soft emission compatible with an accretion shock has also been observed \cite{2010A&A...522A..56G} - again much brighter than in CTTS due to the increased accretion rate. 
On the other hand, in CTTS with their much lower accretion rates, we have no evidence of any correlation between magnetic flares and accretion events \cite{1997A&A...324..155G,2019ApJ...876..121E}.