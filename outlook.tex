
\section{Summary \& Outlook}

X-ray data have shown new, surprising results for the dynamics of material close to pre-main sequence stars, i.e., for accretion and outflow processes. Accretion produces clear observational signatures in the X-ray spectra of young stellar systems, namely cool emission from a high density plasma as shown by XMM-Newton and Chandra grating spectra of the nearest and X-ray brightest systems. Similarly, XMM-Newton and Chandra observations have revealed about one dozen systems with  X-rays from outflows, likely from the systems with the fastest and most powerful jets.
Both processes are strong sources of ionizing radiation and, thus, may impact the entire star formation process including how planets form.

The existing observations, however, suffer from low photon numbers and modest spectral resolution. Day-long integrations are needed to obtain a density estimate for the accretion-generated plasma in the nearest, brightest systems, or to get a temperature estimate for the jet-powered X-rays. We expect that the comparison of the X-ray data with tracers for lower temperature plasma at other wavelengths offers great potential, but the rather long observations currently required for the X-ray spectra pose issues for the intrinsically variable accretion process, because the accompanying, complementary data often provide only snapshots of the accretion state during the X-ray observations. Also, X-ray emitting jets are just too faint to measure line shifts, which would inform us about the process that generates the X-rays in the first place. 

New X-ray facilities like the ESA mission Athena+ will provide the sensitivity to not only enlarge the samples, but will provide those diagnostics that are needed to test models. Measuring the line shifts for the accretion- and jet-powered X-rays will be among the most important tools to decipher the underlying physics as their velocity profiles 
encodes their origin. Even more can be learnt from spectrally resolving the accretion and outflow emission lines. And while Athena+ will be instrumental for pushing our understanding of the hot and energetic processes of young stellar systems, X-ray observatories providing higher resolution spectra have been proposed, too, which are more appropriate for measuring line shapes and not just shifts, like Arcus or Lynx proposed for US missions.

Accretion and outflows are defining characterics of star and planet formation, and both processes genuinely generate X-rays. These X-rays control many of the traditionally studied tracers, e.g., through the heating of the infalling  material or by affecting disk ionization. Any model of star and planet formation must, therefore, also include the most energetic part, the X-ray emission.
