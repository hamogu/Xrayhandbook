
% Around 250 words.
\abstract{Low-mass pre-main sequence stars are strong X-ray sources, because they possess a hot corona like their older main-sequence counterparts. Unique to young stars, however, are X-rays from accretion and outflows, and both processes are of pivotal importance for star and planet formation. We describe how X-ray data provide unique insight into the physics of accretion and outflows. First, mass accreted from a circumstellar disk onto the stellar surface reaches velocities up to a few hundred km/s, fast enough to generate soft X-rays in the post-shock region of the accretion shock. X-ray observations together with laboratory experiments and numerical simulations show that the accretion geometry is complex in young stars. Specifically, the center of the accretion column is likely surrounded by material shielding the inner flow from view but itself also hot enough to emit X-rays. Second, X-rays are observed in two locations of protostellar jets: an inner stationary emission component probably related to outflow collimation and an outer component that evolves withing years and is likely related to working surfaces where the shock travels through the jet. Jet-powered X-rays appear to trace the fastest jet component and  provide unique information on jet launching in young stars. We conclude that X-ray data will continue to be highly important for understanding star and planet formation: they directly probe the origin of many emission features studied in other wavelength regimes. In addition, future X-ray missions will improve sensitivity and spectral resolution to probe key model parameters (e.g. velocities) in large samples of pre-main sequence stars.} 
