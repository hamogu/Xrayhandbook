%%%%%%%%%%%%%%%%%%%% author.tex %%%%%%%%%%%%%%%%%%%%%%%%%%%%%%%%%%%
%
% Template for the Handbook of X-ray and Gamma-ray Astrophysics (preliminary version)
%
%%%%%%%%%%%%%%%% Springer %%%%%%%%%%%%%%%%%%%%%%%%%%%%%%%%%%
\documentclass[graybox, nosecnum]{svmult}

% choose options for [] as required from the list
% in the Reference Guide

\usepackage{mathptmx}       % selects Times Roman as basic font
\usepackage{helvet}         % selects Helvetica as sans-serif font
\usepackage{courier}        % selects Courier as typewriter font
\usepackage{type1cm}        % activate if the above 3 fonts are
                            % not available on your system
%
\usepackage{makeidx}         % allows index generation
\usepackage{graphicx}        % standard LaTeX graphics tool
                             % when including figure files
\usepackage{multicol}        % used for the two-column index
\usepackage[bottom]{footmisc}% places footnotes at page bottom
\usepackage{hyperref}        %for hyperlinks
\usepackage{soul}            % for high-lighting of text
\hypersetup{colorlinks=true,urlcolor=blue}
%
\usepackage[square,numbers]{natbib}
%\bibliographystyle{ieeetr}
\newcommand{\hbindex}[1]{\hl{#1}\index{#1}}  %highlights index entries
\makeindex             % used for the subject index
                       % please use the style svind.ist with
                       % your makeindex program
%%%%%%%%%%%%%%%%%%%%%%%%%%%%%%%%%%%%%%%%%%%%%%%%%%%%%%%%%%%%%%%%%%%%%%%%%%%%%%%%%%%%%%%%%

\begin{document}
%\tableofcontents{}
\title*{Pre main sequence:  Accretion \& Outflows}
% Use \titlerunning{Short Title} for an abbreviated version of
% your contribution title if the original one is too long
\author{Christian P. Schneider  \thanks{corresponding author}, H. Moritz G\"unther and S. Ustamujic}
% Use \authorrunning{Short Title} for an abbreviated version of
% your contribution title if the original one is too long
\institute{Christian P. Schneider \at Institute 1, Address of Institute 1, \email{name1@email.address}
\and H. Moritz G\"unther \at Kavli Institute for Astrophysics and Space Research, Massachusetts Institute of Technology,
77 Massachusetts Avenue, Cambridge, MA 02139, USA \email{hgunther@mit.edu}
\and S. Ustamujic \at Institute 2, Address of Institute 2 \email{sabina.ustamujic@inaf.it}}
%
% Use the package "url.sty" to avoid
% problems with special characters
% used in your e-mail or web address
%
\maketitle
%
\abstract{Each chapter should be preceded by an abstract (about 250 words) that summarizes the content. The abstract will appear \textit{online} at \url{www.SpringerLink.com} and be available with unrestricted access. This allows unregistered users to read the abstract as a teaser for the complete chapter. Please do not include reference citations, cross-references or undefined abbreviations in the abstract.}
\section{Keywords}
Please provide keywords required to facilitate search of chapter on web; maximum 10 keywords.


\section{Introduction}
(3 pages) [Christian]

Star formation sketch [Christian adapted from McCaughrean]
\subsection{Relevant History of T Tauri Stars}
Sketch of T Tauri system [Christian]
\subsection{Properties of T Tauri star systems}
(disk, accretion, outflows)
\subsection{Brief context of other relevant observations}
(Ha for accretion, images/IFU for jets)

Examplary Xshooter spectrum [HMG: I sugest to just ask Carlo for a figure] Venuti et al. 2019?

        IFU image(s) from Takami monitoring

\section{Accretion}
(10 pages)
\subsection{Physics of accretion}
(free-fall, accretion shock, plasma temperature, X-ray emission, etc.) [Moritz, Sabina]

Accretion sketch Hartmann et al. 2016?

\subsection{Observations of Accretion signatures}
Moritz, Christian

Kastner & Brickhouse spectra, other spectra?, Argiroffi redshift? yes, Brickhouse 2012 variability, probably also want some Ha variability, even if we write an X-ray review. Large choice, maybe Alencar?

\subsection{Models}
(historic evolution of models and aspects addressed by models) [Moritz, (Sabina)]

Accretion column models Lamzin 2004, Robinson & Espaillat?; C IV lines (Lamzin 2003?); accretion rates

a figure from any Romanovy work


Radiative transfer (Colombo et al. )

Laboratory experiments (e.g. Burdonov et al. 2020)



\subsection{Connection to other observations}
           Moritz

 Observations of the magnetospheric accretion region (e.g Bouvier et al.)


\section{Jets}
         (10 pages)
\subsection{Physics of jets}
(jet origin, collimation, jet velocities) [Christian, Sabina]

Sketch (suggestion?) should we make a new sketch?

Wind, outflows and jets: Bally 2016 (main components of outflows); Matt & Pudritz 2005 (accretion powered winds), Zanni & Ferreira 2013 (magnetospheric ejections)


\subsection{X-ray jet observations}
Christian

Favata HH 154, Guedel DG Tau, Schneider HH 154

Laboratory experiments  e.g. Revet et al. 2017

\subsection{Models of X-ray emission from jets}
Sabina

3D view based on Ustamujic et al. MHD model

Stationary X-ray emission, HH objects

Collimation by the rotating magnetic field or wind pressure

Bonito et al? (HMG: Not a big fan of the setup wit ha ``nozzle'' at the disk which then leads to the diamond shock, but they are often cited and instructuve picture)


Lab work in here, too?

\subsection{Connection to other obs}
Christian

UV obs (DG Tau jet), [Fe II] monitoring, Liu [Ne III]?

Jet rotation

\section{Outlook}
(2 pages) [all]

Athena, Lynx, Arcus (figures will be new, Moritz can make for Lynx, Arcus)(?)



\section{Instruction - DELETE}

\vspace{0.3cm}

{\bf Key-points to have in mind}:
\begin{enumerate}
\item The {\it Handbook of X-ray and Gamma-ray Astrophysics} is aiming to publish a work of tertiary literature, which provides easily accessible, digested and established knowledge derived from primary or secondary sources in the particular field. Therefore, your contribution should be clear and concise and be a comprehensive and up-to-date overview of your topic.
\item Length of text: every chapter normally consists of about 10,000-20,000 words (excluding figures and references), which would be about 20-40 typeset pages. However, this is not a rigid rule and it is something to discuss with the Section Editors of the Section of the chapter.
\item Colored figures are welcome in any standard format (jpg, tif, ppt, gif). If possible please provide the original figure in high resolution (300 dpi minimum). {\color{red}\bf Please do not forget to obtain permission in case the figure is from a published article}. For journals like ApJ, PRD, PRL, etc. you do not need the permission if you are an author of the article of the original figures, but for most journals you need the permission even if you are the author of those figures. However, if you create a new figures with minor changes, you can claim it is a new figure and you do not need any permission.
\end{enumerate}


\vspace{0.5cm}

{\bf Submission}:

\vspace{0.1cm}

{\color{red} \bf Please submit source files, compiled pdf file of the chapter, and figure permissions through the Meteor system.}

\vspace{0.5cm}

{\bf arXiv policy}:

\vspace{0.1cm}

{Authors can post their chapter on arXiv if they wish to do it. In the comment field, please write something like ``Invited chapter for {\it Handbook of X-ray and Gamma-ray Astrophysics} (Eds. C. Bambi and A. Santangelo, Springer Singapore, expected in 2022)''. After the publication of the chapter, you may add the DOI on the arXiv page.}



\section{Cross-References \textit{(if applicable)}}
Include a list of related entries from the handbook here that may be of further interest to the readers.

% bibs will be alphametically ordered. I think that's what they want.
% if not, see: https://tex.stackexchange.com/questions/312819/problem-with-springer-bibliography-stylespbasic-natbib
\bibliographystyle{spbasic}
\bibliography{bib.bib}

\end{document}
